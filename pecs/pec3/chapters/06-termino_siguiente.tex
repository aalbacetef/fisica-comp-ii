\section{Problema F - Regla del término siguiente}
Utilizando la regla del término siguiente estime el error cometido al datar la muestra cuya proporción de Carbono-14 es $0.8705$ con las aproximaciones calculadas en el apartado C) y D).

\subsection{Análisis}

La regla del término siguiente se basa en aproximar el error definiéndolo como la diferencia entre el polinomio de orden $n$ y el polinomio de orden $m = n + 1$.

Básicamente es el valor del término $m = n + 1$.


$$ E_{sig}(x)_n = P_{n+1}(x) - P_{n}(x) $$

\subsection{Resolución}

\subsubsection{Programación}

Utilizaremos el siguiente código para calcular el error:

\lstinputlisting[language=Python]{../../code/pecs/pec3/ex6.py}

\newpage 

\subsection{Resolución}

Los errores de cada orden, para el valor de interés $0.8705$ se han calculado:

\begin{table}[htbp]
	\centering
	\csvreader[
	tabular=|c|c|c|,
	table head=\hline \textbf{Polynomial} & \textbf{Error} \\\hline,
	late after last line=\\\hline,
	]{data/errors05.csv}{}{\csvlinetotablerow}
\end{table}

