\section{Problema E - Error absoluto cometido}

\paragraph{Ecuación}
La antigüedad de la muestra es dada por:

\begin{equation}
	t = - \ln (N_{14C}) \frac{t_{12}}{\ln 2}
\end{equation}

y $t = 5730$



Estime el error absoluto cometido con las aproximaciones obtenidas en los 
apartados A) y C) y D) en relación a la edad t dada por expresión anterior. 

Realice una discusión sobre los órdenes de magnitud de los errores calculados.

\subsection{Resolución}

\subsubsection{Programación}
Si usamos la ecuación, vemos que la edad es: $1146.480157$.

Utilizaremos el siguiente código para calcular el error:

\lstinputlisting[language=Python]{../../code/pecs/pec3/ex5.py}

\newpage 

\subsubsection{Resultados}

Con esto tenemos los resultados en la siguiente tabla:

\begin{table}[htbp]
	\centering
	\csvreader[
	tabular=|c|c|c|,
	table head=\hline \textbf{Polynomial} & \textbf{Age} & \textbf{Error} \\\hline,
	late after last line=\\\hline,
	]{data/errors04.csv}{}{\csvlinetotablerow}
\end{table}

\subsection{Discusión}

Como se puede ver, el error es dos órdenes de magnitud por debajo del valor, lo cual es una aproximación relativamente buena.