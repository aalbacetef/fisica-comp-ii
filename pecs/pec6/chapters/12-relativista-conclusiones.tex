\section{Ejercicio GR - Conclusiones}

\subsection{Problema}

Extraiga las conclusiones de todos los resultados del trabajo (parte no relativista y relativista).


\subsection{Discusión sobre la realización del trabajo}

Quizás lo que más me ha sorprendido de este trabajo, y que consecuentemente ha sido de gran aprendizaje, es lo potente que es combinar el método RK4 con la capacidad de procesado de datos que aporta el PC. 


\paragraph{Posibles errores}
En las dos partes, lo más complicado ha sido hacer a mano los cálculos para expresar el problema como un sistema de ecuaciones diferenciales de primer orden, teniendo mucho cuidado con los signos y demás detalles (e.g: el haberme equivocado y escrito un '+' donde tenía que haber un '-' significó pasarme unas horas buscando un bug cuando el error no estaba en el código...).

También se me hizo muy claro la utilidad de hacer tests sobre el algoritmo, para asegurarme que la implementación del método fue correcta (e.g: para asegurarme que implementé el algoritmo RK4 de forma correcta hice algunos tests donde $\vec{y}(t) = [\sin wt, \cos wt]$, y comprobé los resultados de una simulación con los valores teóricos). Esto ayuda muchísimo cuando estás generando datos incorrectos y tienes que ver si los errores están en el algoritmo, en el sistema calculado a mano, o en el traspaso del sistema de papel a código.

\paragraph{Visualización e interpretación}

Noté lo díficil que puede llegar ser extraer una interpretación de los datos, pues simular es sólo una parte del problema, luego hay que visualizar e interpretar los resultados.  

\subsection{Discusión sobre los resultados}

\subsubsection{Parte no relativista} En la parte no relativista, hicimos una simulación de un ciclotrón, donde los resultados mostraban a la partícula haciendo una trayectoria de espiral, avanzando principalmente en la dirección del eje-x. 

\subsubsection{Parte relativista} Aquí, al visualizar el módulo de la velocidad, así como la velocidad radial y del componente $v_z$, se puede ver que son valores constantes. Cosa poco sorprendente pues no hay campo $E$ y el campo $B$ es no-uniforme pero estático, por ende no hace trabajo sobre la partícula.  

\paragraph{Error numérico}No me he puesto a hacer cálculos para comprobar esto, pero es muy posible que los valores de los componentes y sean producto de un error numérico, dada la descomunal diferencia en el orden de magnitud entre el componente-y con los componentes x y z de la velocidad y la posición. Quizás modificando el algoritmo para usar sumas compensadas (e.g: Kahan) minimice la presencia de estos errores (aunque hay más fuentes de errores). 