\section{Ejercicio C - Como aplicar el método RK4}

\subsection{Problema}

Describa detalladamente como aplicar el método de Runge-Kutta de orden cuatro al sistema de ecuaciones diferenciales ordinarias del apartado previo. En concreto, obtenga explícitamente las expresiones funcionales de los coeficientes de Runge-Kutta implicados en la resolución.

\subsection{Análisis}

El método de Runge-Kutta de orden 4, en un contexto vectorial, está definido para un problema de valor inicial:

$$
\diff{\vec{y}}{t} = \vec{f}(t, \vec{y})
$$

con $\vec{y}(t_0) = \vec{y_0}$. 

Así que lo primero es replantear el sistema del apartado anterior: 


\begin{align*}
	q v_y - m \dot{v_x} &= 0 \\
	q(1 - v_x) - m \dot{v_y} &= 0 \\
	-m \dot{v_z} &= 0
\end{align*}

para que tenga la misma forma. Podemos reajustar y tener:

\begin{align*}
	\dot{v_x} &= \frac{q}{m} v_y \\
	\dot{v_y} &= \frac{q}{m}(1 - v_x) \\
	\dot{v_z} &= 0
\end{align*}

Lo cual da que:

$$
\vec{y}(t) = \vec{v}(t)
$$

$$
\vec{f}(t, \vec{v}) = \frac{q}{m} \begin{bmatrix}
\vec{v}(t) \cdot \uvec{j} \\ \\
1-\vec{v}(t) \cdot \uvec{i}\end{bmatrix}
$$

donde he usado los producto-punto en vez de $v_x(t)$ y $v_y(t)$ para ser explícito.

\newpage 

\subsection{Resolución}

Los 4 coeficientes son, por $i = 0 ... n$.

\begin{align*}
	\vec{k}_1 &=
h\vec{f}(t_i, \vec{y}_i) \\
	\vec{k}_2 &=
h\vec{f}(
	t_i + \frac{h}{2}, 
	\vec{y}_i + \frac{\vec{k_1}}{2}
) \\
	\vec{k}_3 &=
h\vec{f}(
	t_i + \frac{h}{2}, 
	\vec{y}_i + \frac{\vec{k}_2}{2}
) \\
	\vec{k}_4 &= 
h\vec{f}(
	t_i + h,
	\vec{y}_i + \vec{k}_3
)
\end{align*}

En cada paso: 

\begin{align*}
	t_{i+1} &= t_i + h \\
	\vec{y}_{i+1} &=
\vec{y}_i + \frac{1}{6}(
	\vec{k}_1 +
	2\vec{k}_2 +
	2\vec{k}_3 +
	\vec{k}_4
)
\end{align*}

\paragraph{Código} El código que implementa el método RK4 se puede ver en \ref{code:rk4}.