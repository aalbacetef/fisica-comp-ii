\section{Ejercicio A - La segunda ley de Newton}

\subsection{Problema}

Plantee la segunda ley de Newton para la partícula cuando el campo eléctrico y magnético son ambos uniformes, constantes y perpendiculares entre sí. Simplifique para el caso particular $vec{E} = (0, 1, 0) \frac{V}{m}$, $vec{B} = (0, 0, 1)\ T$.

\subsection{Análisis}

La segunda ley de Newton define la fuerza sobre un cuerpo como:

\begin{equation}\label{eq:1}
	\vec{F} = m \vec{a}
\end{equation}

\subsection{Resolución}

La fuerza de Lorentz sobre una partícula de masa $m$ y carga $q$ es:

\begin{equation}
	\vec{F} = q(\vec{E} + \vec{v} \times \vec{B})
\end{equation}

En el caso de que $\vec{E} = (0, 1, 0) \frac{V}{m}$, $\vec{B} = (0, 0, 1)\ T$:

\begin{align*}
\vec{F}(t)  
& =
	q(\uvec{j} + \vec{v}(t) \times \uvec{k}) \\
& =
	q(\uvec{j} + v_y \uvec{i} - v_x \uvec{j} ) \\
& = 
	qv_y \uvec{i} + q(1-v_x)\uvec{j} \\
& =
	q \begin{bmatrix}
		v_y \\
		1 - v_x \\
		0
	\end{bmatrix}
\end{align*}

Igualando esto con \eqref{eq:1}, nos da:

\begin{equation*}
	q \begin{bmatrix}
	v_y \\
	1 - v_x \\
	0
\end{bmatrix} = m \vec{a}(t)
\end{equation*}