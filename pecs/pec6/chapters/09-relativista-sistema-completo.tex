\section{Ejercicio DR - Sistema de ED Completo}

\subsection{Problema}
Presente el sistema de ecuaciones diferenciales completo que hay que resolver para describir el movimiento de una partícula cargada bajo la acción simultánea de un campo eléctrico y magnético. ¿Cuántas ecuaciones diferenciales de primer orden es preciso resolver?

\subsection{Resolución}

El sistema completo consiste en las siguientes ecuaciones:

\begin{equation*} 
	\begin{bmatrix} 
	\diff{r_x}{t} \\ \\
	\diff{r_y}{t} \\ \\ 
	\diff{r_z}{t} \\ \\
	\diff{p_{rel_x}}{t} \\ \\
	\diff{p_{rel_y}}{t} \\ \\ 
	\diff{p_{rel_z}}{t} 
\end{bmatrix} = 
\begin{bmatrix}
	p_{rel_x}(t) \\ \\ p_{rel_y}(t) \\ \\ p_{rel_z}(t)
\\ \\
		\frac{q}{\sqrt{ 
			m^2 + \frac{1}{c^2} \sum p_{rel_k}	
	}} \left( 
	(\sqrt{ 
		m^2 + \frac{1}{c^2} \sum p_{rel_k}	
	})\begin{bmatrix} 
		E_x \\ E_y \\ E_z 
	\end{bmatrix} + 
	\begin{bmatrix} 
		p_{rel_x} \\ p_{rel_y} \\ p_{rel_z}
	\end{bmatrix} 
	\times 
	\begin{bmatrix} B_x \\ B_y \\ B_z \end{bmatrix} 
	\right)
\end{bmatrix} 
\end{equation*}

En total 6 ecuaciones de primer orden.