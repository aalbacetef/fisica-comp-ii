\section{Ejercicio ER - Sistema de ED Completo}

\subsection{Problema}

Utilizando un Runge-Kutta de cuarto orden con un paso temporal de $\Delta t = 10^{-12} s$ realice la simulación necesaria para describir el movimiento del electrón cuando parte de las condiciones iniciales

\begin{align*}
	\vec{r}(0) &= \gamma m c  (-1, 1, 0 )\ \text{m} \\
	\vec{v}(0) &= \frac{2c}{3} 
		(
			\sin \frac{\pi}{9},
			0,
			\cos \frac{\pi}{9} 
		) \ \text{m\slash s}
\end{align*} 

hasta un tiempo total $t = 200\ \text{ns}$. Indique los resultados obtenidos para todas las magnitudes (posiciones y momentos de la partícula) calculadas con el Runge-Kutta en las dos primeras iteraciones temporales.

\subsection{Resolución} 

El código que resuelve este ejercicio se puede ver en \ref{code:ex10}.

Los resultados completos se pueden ver en el repositorio git y los primeros 50 puntos están en \ref{data:r_v_rel}. 

Los valores para las dos iteraciones temporales son:

\begin{table}[H]
	\centering
	\csvreader[
	tabular=|c|l|l|l|,
	table head=\hline \textbf{t} & \textbf{$r_x$} & \textbf{$r_y$} & \textbf{$r_z$} \\\hline,
	filter test=\ifnumless{\thecsvrow}{3},
	late after last line=\\\hline,
	]{data/simul_rel_r_capped.csv}{}{\csvlinetotablerow}
\end{table}

\begin{table}[H]
	\centering
	\csvreader[
	tabular=|c|l|l|l|,
	table head=\hline \textbf{t} & \textbf{$v_x$} & \textbf{$v_y$} & \textbf{$v_z$} \\ \hline,
	filter test=\ifnumless{\thecsvrow}{3},
	late after last line=\\\hline,
	]{data/simul_rel_v_capped.csv}{}{\csvlinetotablerow}
\end{table}


