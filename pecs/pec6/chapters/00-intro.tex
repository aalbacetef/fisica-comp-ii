%
% Introducción
%

\section{Introducción}

\paragraph{}
En este documento se encuentran redactadas las respuestas a la Prueba de Evaluación Continua 6. Se ha utilizado Python como lenguaje de programación y LaTeX para generar este documento.

\subsection{Sobre este documento}

\subparagraph{Estructura}
A cada pregunta se le ha dedicado una sección, en la que se intenta responder a los distintos puntos de la cuestión así como una explicación del código relacionado a la solución de dicha pregunta. En algunos casos, si una pregunta ya ha sido contestada en un apartado anterior, lo anotaré. 

\subparagraph{Código}
El código para esta (y otras PECs) lo estaré publicando en un repositorio git. Se puede acceder via:

\begin{lstlisting}[language=bash]
	git clone https://gitlab.com/aalbacetef/fisica-comp-II.git entrega-aalbacetef-fc-ii
\end{lstlisting}


Si se desea, puedo enviar por correo electrónico el código del proyecto en archivo comprimido (tar/rar/zip/7z/etc...).

Gran parte del código lo he puesto en el apéndice, pero recomiendo ver el repositorio git para poder leerlo más cómodamente.


\subsection{Ejecutar el código}

Para ejecutar el código es necesario tener Python instalado. He facilitado esto con el uso de Docker y un Makefile. Las instrucciones de como ejecutar se pueden encontrar en el README.md del repositorio.

\subsection{Información de contacto}

Si necesita contactarme por alguna razón, aparte de mi correo electrónico de la UNED, puede contactarme mediante:
\begin{itemize}
	\item \textbf{Email:} aalbacetef@gmail.com
\end{itemize}

\subsection{Afirmación de autoría del trabajo}

\paragraph{}

El firmante de este trabajo reconoce que todo él es original, de su única autoría, escritura
y redacción, y que allí donde han sido empleadas ideas o datos de otros autores, su
trabajo ha sido reconocido y ubicado, con suficiente detalle, como para que el lector
pueda consultar lo afirmado sobre él.