\section{Ejercicio AR - Sistema de ED}

\subsection{Problema}

Utilizando coordenadas cartesianas en el espacio tridimensional para todas las magnitudes vectoriales, escriba el sistema de ecuaciones diferenciales pervio en términos de los componentes $x$, $y$, $z$ de $\vec{p}$, $\vec{v}$, $\vec{E}$ y $\vec{B}$.

La velocidad depende del momento a través de un factor $\gamma m$. Reescriba ese sistema de ecuaciones diferenciales en términos de los componentes del momento, dejando presente explícitamente el factor $\gamma m$.

\subsection{Resolución}

El sistema de ecuaciones dado por la fuerza de Lorentz es:

\begin{equation*}
\begin{bmatrix} 
	\diff{p_{rel_x}}{t} \\ \\
	\diff{p_{rel_y}}{t} \\ \\ 
	\diff{p_{rel_z}}{t} 
\end{bmatrix} = 
q \left( 
\begin{bmatrix} 
	E_x \\ E_y \\ E_z 
\end{bmatrix} + 
\begin{bmatrix} 
	v_x \\ v_y \\ v_z 
\end{bmatrix} 
	\times 
\begin{bmatrix} B_x \\ B_y \\ B_z \end{bmatrix} 
\right)
\end{equation*}

donde usamos $p_{rel}$ para ser explícito en que es el momento relativista.

La velocidad se puede reescribir en términos del momento: $ \vec{v} = \frac{1}{\gamma m} \vec{p}_{rel} $.

Por ende:

\begin{equation*}
	\begin{bmatrix} 
		\diff{p_{rel_x}}{t} \\ \\
		\diff{p_{rel_y}}{t} \\ \\ 
		\diff{p_{rel_z}}{t} 
	\end{bmatrix} = 
	q \left( 
	\begin{bmatrix} 
		E_x \\ E_y \\ E_z 
	\end{bmatrix} + 
\frac{1}{\gamma m}
	\begin{bmatrix} 
		p_{rel_x} \\ p_{rel_y} \\ p_{rel_z}
	\end{bmatrix} 
	\times 
	\begin{bmatrix} B_x \\ B_y \\ B_z \end{bmatrix} 
	\right) =
	\frac{q}{\gamma m} \left( 
(\gamma m)\begin{bmatrix} 
	E_x \\ E_y \\ E_z 
\end{bmatrix} + 
\begin{bmatrix} 
	p_{rel_x} \\ p_{rel_y} \\ p_{rel_z}
\end{bmatrix} 
\times 
\begin{bmatrix} B_x \\ B_y \\ B_z \end{bmatrix} 
\right)
\end{equation*}


