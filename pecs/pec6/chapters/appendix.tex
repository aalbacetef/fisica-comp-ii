\section{Apéndice} 

\subsection{Tipos}\label{code:types}

\lstinputlisting[language=Python]{../../code/methods/types.py}
\newpage

\subsection{Util}\label{code:util}

\lstinputlisting[language=Python]{../../code/util.py}
\newpage

\subsection{Datos}\label{code:data}

Aquí he definido constantes y funciones utilizadas por varios ejercicios.

\lstinputlisting[language=Python]{../../code/pecs/pec6/data.py}
\newpage


\subsection{Código linalg}
\label{code:linalg}

Las funciones nuevas son (están al final): 
\begin{itemize}
	\item vec\_sum
	\item vec\_scalar
	\item vec\_copy
	\item curl
\end{itemize}

\lstinputlisting[language=Python]{../../code/methods/linalg.py}
\newpage

\subsection{Código RK4}
\label{code:rk4}

\lstinputlisting[language=Python]{../../code/methods/runge_kutta.py}
\newpage

\subsection{Código Ejercicio D}
\label{code:ex4}

\lstinputlisting[language=Python]{../../code/pecs/pec6/ex4.py}
\newpage

\subsection{Datos RK4 - $\vec{r}(t)$ y $\vec{v}(t)$ no relativista (primeros 50 puntos)}
\label{data:r_v_nonrel}

\begin{table}[H]
	\centering
	\csvreader[
	tabular=|c|l|l|l|,
	table head=\hline \textbf{t} & \textbf{$r_x$} & \textbf{$r_y$} & \textbf{$r_z$} \\\hline,
	filter test=\ifnumless{\thecsvrow}{50},
	late after last line=\\\hline,
	]{data/simul_no_rel_r.csv}{}{\csvlinetotablerow}
\end{table}

\begin{table}[H]
	\centering
	\csvreader[
	tabular=|c|l|l|l|,
	table head=\hline \textbf{t} & \textbf{$v_x$} & \textbf{$v_y$} & \textbf{$v_z$} \\ \hline,
	filter test=\ifnumless{\thecsvrow}{50},
	late after last line=\\\hline,
	]{data/simul_no_rel_v.csv}{}{\csvlinetotablerow}
\end{table}

\subsection{Código Ejercicio E}
\label{code:ex5}

\lstinputlisting[language=Python]{../../code/pecs/pec6/ex5.py}
\newpage

\subsection{Código Ejercicio ER}
\label{code:ex10}

\lstinputlisting[language=Python]{../../code/pecs/pec6/ex10.py}
\newpage



\subsection{Datos RK4 - $\vec{r}(t)$ y $\vec{v}(t)$ relativista (primeros 50 puntos)}
\label{data:r_v_rel}

\begin{table}[H]
	\centering
	\csvreader[
	tabular=|c|l|l|l|,
	table head=\hline \textbf{t} & \textbf{$r_x$} & \textbf{$r_y$} & \textbf{$r_z$} \\\hline,
	late after last line=\\\hline,
	]{data/simul_rel_r_capped.csv}{}{\csvlinetotablerow}
\end{table}

\begin{table}[H]
	\centering
	\csvreader[
	tabular=|c|l|l|l|,
	table head=\hline \textbf{t} & \textbf{$v_x$} & \textbf{$v_y$} & \textbf{$v_z$} \\ \hline,
	late after last line=\\\hline,
	]{data/simul_rel_v_capped.csv}{}{\csvlinetotablerow}
\end{table}


\subsection{Código Ejercicio FR}
\label{code:ex11}

\lstinputlisting[language=Python]{../../code/pecs/pec6/ex11.py}
\newpage

