\section{Ejercicio D - RK4 para un electrón}

\subsection{Problema}

Sea la partícula un electrón, es decir, con carga $q = -1.602 \cdot 10^{-19} C$  y masa $m = 9.109 \cdot 10^{-31} kg$. Realice la resolución numérica del problema con el método de Runge-Kutta de orden cuatro para los tiempos $t \in [0 \text{s}, 2\cdot10^{-10} \text{s} ]$ utilizando un paso de tiempo $\Delta t = 1 \cdot 10^{-12} \text{s}$ y con las condiciones iniciales de la posición $r(t = 0) = (0, 0, 0) \text{m}$ y velocidad inicial $v(t=0) = (0, 1, 0) \frac{\text{m}}{\text{s}}$.  Indique los resultados obtenidos para todas las magnitudes (posiciones y velocidades de la partícula) calculadas con el Runge-Kutta en las dos primeras iteraciones temporales.

\subsection{Resolución}

El código que resuelve el ejercicio se puede ver en \ref{code:ex4}.

Los resultados completos se pueden ver en el repositorio git y los primeros 50 puntos están en \ref{data:r_v_nonrel}. 

Los valores para las dos iteraciones temporales son:


\begin{table}[H]
	\centering
	\csvreader[
	tabular=|c|l|l|l|,
table head=\hline \textbf{t} & \textbf{$r_x$} & \textbf{$r_y$} & \textbf{$r_z$} \\\hline,
	filter test=\ifnumless{\thecsvrow}{3},
	late after last line=\\\hline,
	]{data/simul_no_rel_r.csv}{}{\csvlinetotablerow}
\end{table}

\begin{table}[H]
	\centering
	\csvreader[
	tabular=|c|l|l|l|,
	table head=\hline \textbf{t} & \textbf{$v_x$} & \textbf{$v_y$} & \textbf{$v_z$} \\ \hline,
	filter test=\ifnumless{\thecsvrow}{3},
	late after last line=\\\hline,
	]{data/simul_no_rel_v.csv}{}{\csvlinetotablerow}
\end{table}


