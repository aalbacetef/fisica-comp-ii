\section{Ejercicio BR - Factor Lorentz}

\subsection{Problema}
Exprese el valor del factor $\gamma m$ en términos de los componentes del momento. Finalmente, utilice esa expresión para reescribir el sistema de ecuaciones diferenciales en términos de los componentes cartesianos del momento e indique si se trata de un sistema de ecuaciones diferenciales acopladas o no.

\subsection{Resolución}

Empezamos por:

\begin{equation*}
	\gamma = 
\frac{1}{
	\sqrt{
	1 - \frac{|| \vec{v} ||^2} {c^2}	
}
}
\end{equation*}

podemos obtener una ecuación más fácil de manipular si:

\begin{align*}
	\gamma \sqrt{ 1 -  \frac{||\vec{v}||^2}{c^2}} &= 1 \\
	\gamma^2 (1 - \frac{||\vec{v}||^2}{c^2}) &= 1 \\
	\gamma ^2 - \frac{\gamma^2}{c^2} || \vec{v} ||^2 & =1
\end{align*}

Ahora reescribimos $||\vec{v}||^2$ en términos del momento relativista:

\begin{align*}
|| \vec{v} ||^2 
	&= \vec{v}\cdot \vec{v} \\
	&= \sum v_k^2 \\
	&= \sum (\frac{p_{rel_k}}{\gamma m})^2 \\
	&= \frac{1}{(\gamma m)^2} \sum p_{rel_k}^2 \\
\end{align*} 

Lo que finalmente da:

\begin{align*}
\gamma ^2 - \frac{\gamma^2}{c^2} || \vec{v} ||^2  &= 1 \\
\gamma ^2 - \frac{\gamma^2}{c^2} \frac{1}{(\gamma m)^2} \sum p_{rel_k}^2  &= 1
\end{align*}

Ahora podemos obtener una expresión para $\gamma m$:

\begin{align*} 
\gamma ^2 &=
1 + \frac{1}{(mc)^2} \sum p_{rel_k}^2 \\
m^2 \gamma^2 &= 
	m^2 + \frac{1}{c^2} \sum p_{rel_k} \\
m \gamma &=
	\sqrt{ 
	m^2 + \frac{1}{c^2} \sum p_{rel_k}	
}
\end{align*} 

Empezando por la última ecuación del ejercicio anterior:

\begin{equation*} 
\begin{bmatrix} 
	\diff{p_{rel_x}}{t} \\ \\
	\diff{p_{rel_y}}{t} \\ \\ 
	\diff{p_{rel_z}}{t} 
\end{bmatrix} = 
	\frac{q}{\gamma m} \left( 
(\gamma m)\begin{bmatrix} 
	E_x \\ E_y \\ E_z 
\end{bmatrix} + 
\begin{bmatrix} 
	p_{rel_x} \\ p_{rel_y} \\ p_{rel_z}
\end{bmatrix} 
\times 
\begin{bmatrix} B_x \\ B_y \\ B_z \end{bmatrix} 
\right)
\end{equation*} 

\begin{equation*} 
	\begin{bmatrix} 
		\diff{p_{rel_x}}{t} \\ \\
		\diff{p_{rel_y}}{t} \\ \\ 
		\diff{p_{rel_z}}{t} 
	\end{bmatrix} = 
	\frac{q}{\sqrt{ 
			m^2 + \frac{1}{c^2} \sum p_{rel_k}	
	}} \left( 
	(\sqrt{ 
		m^2 + \frac{1}{c^2} \sum p_{rel_k}	
	})\begin{bmatrix} 
		E_x \\ E_y \\ E_z 
	\end{bmatrix} + 
	\begin{bmatrix} 
		p_{rel_x} \\ p_{rel_y} \\ p_{rel_z}
	\end{bmatrix} 
	\times 
	\begin{bmatrix} B_x \\ B_y \\ B_z \end{bmatrix} 
	\right)
\end{equation*} 

Evidentemente estamos ante un sistema de ecuaciones diferenciales acopladas, ya que la ecuación diferencial de cada componente del momento depende de los otros componentes. Esto no sólo es debido a $\gamma$ sino también por $\vec{p}_{rel} \times \vec{B}$ (en el caso de $\vec{B}$ no nulo).