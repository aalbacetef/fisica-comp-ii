\section{Error en la aproximación de Padé}

\subsection{Problema}
En relación a la función de ajuste $D(t)$ calculada por mínimos cuadrados, obtenga la funcióde error absoluto que se comete con el aproximante de Padédel apartado a) calculado alrededor de tres puntos distintos $t_0 = 1.0$, $t_0 = 1.5$ y $t_0 = 2.0$ y para los valores $\alpha$ y $\beta$ calculados en el ajuste. Calcule el error cuadrático en todos los casos y discuta la bondad de los ajustes.


\subsection{Resolución}

El código que se ha usado en la resolución de este problema está en \ref{code:ex6}. Se puede ver gráficamente el error absoluto de las tres aproximaciones en respecto al ajuste de mínimos cuadrados.

\begin{figure}[H]
	\includegraphics[width=\linewidth]{figures/error_abs_pade.png}
	\caption{Error de las aproximaciones de Padé en respecto al ajuste de mínimos cuadrados.}
	\label{fig:error_abs_pade}
\end{figure}
