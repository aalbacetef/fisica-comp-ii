\section{Serie de Taylor}

\subsection{Problema}
Obtenga la expresión del desarrollo de Taylor de tercer grado para la función $D(t)$
dada al inicio del enunciado alrededor de un t0 arbitrario. Particularice a los valores 
$t0 = 1.5$ y $\alpha$, $\beta$ calculados en el ajuste.

Calcule el error cuadrático que cometemos con el desarrollo de Taylor a la hora
de representar la tendencia de los datos experimentales.

Con respecto a la mejor función de ajuste $D(t)$ calculada por mínimos cuadrados,
calcule dos funciones de error: el error absoluto que se comete con este desarrollo de
Taylor y el error absoluto que se comete con el aproximante de Padé desarrollado
en torno a $t0 = 1.5$. Represente en una misma gráfica estas funciones de error y
extraiga las conclusiones.

\subsection{Resolución}

Por suerte, ya hemos calculado estos coeficientes en el apartado a).

\begin{align*}
	&a_0 = D(t=t_0)  = e^{-\alpha t_0} + \beta \sin(t_0) \\
&a_1 = \frac{\partial_t D(t=t_0)}{1!} 
= -\alpha e^{-\alpha t_0} + \beta \cos(t_0) \\
&a_2 = \frac{\partial_t^2 D(t=t_0)}{2!}
= \frac{\alpha^2 e^{-\alpha t_0} - \beta \sin(t_0)}{2}
\\
&a_3 = \frac{\partial_t^3 D(t=t_0)}{3!}
= \frac{-\alpha^3 e^{-\alpha t0} - \beta \cos(t_0)}{6}	
\end{align*}

y tenemos que la serie de Taylor de tercer grado es: 

$$
\sum_{k=0}^{3} a_k (t - t_0)^k
$$