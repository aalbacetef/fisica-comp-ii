\section{Mínimos cuadrados}

\subsection{Problema}

Estamos interesados en ajustar esos datos experimentales al modelo teórico
dado por $D(t)$. Utilice el método de los mínimos cuadrados para obtener el sistema
de ecuaciones que deben satisfacer los parámetros $\alpha$ y $\beta$.

\subsection{Análisis}

El ajuste por mínimos cuadrados, en resumen, se trata de buscar los parámetros que minimicen suma cuadrada de los residuos. 

Si definimos:

\begin{equation}
	r_k = y_k - f(x_k)
\end{equation}

\begin{equation}
	\text{SSR}(\alpha, \beta) = \sum_{k} r_k^2 
\end{equation}

Buscamos parámetros $\alpha$ y $\beta$ que minimicen el SSR. Esto es:

\begin{equation}
	\partial_\alpha \text{SSR} = 0
\end{equation}
\begin{equation}
	\partial_\beta \text{SSR} = 0
\end{equation}

Expandiendo:

\begin{align*}
	\partial_\alpha \text{SSR} &= \partial_\alpha \sum_{k} r_k^2 \\
	&= \sum_{k} \partial_\alpha r_k^2 \\
	&= 2 \sum_{k} r_k  \partial_\alpha r_k 
\end{align*} 

\begin{align*}
	\partial_\beta \text{SSR} &= \partial_\beta \sum_{k} r_k^2 \\
	&= \sum_{k} \partial_\beta r_k^2 \\
	&= 2 \sum_{k} r_k \partial_\beta r_k 
\end{align*} 

y las derivadas parciales son, dado que $\partial_\theta r_k = - \partial_\theta f(x_k)$:

\begin{equation}
	\partial_\alpha f(x_k) = \partial_\alpha (e^{-\alpha x_k} + \beta \sin(x_k)) \\ 
	= -x_k e^{-\alpha x_k}
\end{equation}
\begin{equation}
	\partial_\beta f(x_k) = \partial_\beta (e^{-\alpha x_k} + \beta \sin(x_k)) \\ 
	= \sin(x_k)
\end{equation}

Asi que:

\begin{align*}
	\partial_\alpha \text{SSR} &= 
	2 \sum_{k} r_k(-\partial_\alpha f(x_k)) \\
	&= 2 \sum_{k} r_k(x_k e^{-\alpha x_k}) \\
	&= 0
\end{align*}

\begin{align*}
	\partial_\beta \text{SSR} &= 
	2 \sum_{k}r_k(-\partial_\beta f(x_k)) \\
	&= 2 \sum_{k} r_k (-\sin(x_k)) \\
	&= 0
\end{align*}

