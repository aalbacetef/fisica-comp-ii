\section{Aproximante de Padé}


\subsection{Problema}

Sea la función no lineal definida en términos de dos parámetros $\alpha$ y $\beta$

\begin{equation}
	D(t) = e^{-\alpha t} + \beta sin(t) 
\end{equation}

Calcule el aproximante de Padé de la función $D(t)$ alrededor de $t_0$, con grados dos en el numerador y uno en el denominador. Se recomienda escribir el aproximante como:

$$
\frac{a_0 + a_1 ( t - t_0) + a_2(t - t_0)^2 }{ 1 + b_1(t - t_0) }
$$


\subsection{Análisis}


\subsubsection{Expansiones de Taylor y MacLaurin}

El aproximante de Padé se puede considerar como una extensión de la expansión de Taylor. 

La expansión de Taylor consiste en construir una serie de potencias en grado $n$ tal que:

$$
T_n^{\alpha}(x) = \sum_{k = 0}^{n} a_k (x - \alpha)^k 
$$

Y la serie de MacLaurin es la expansion $\alpha = 0$, asi que:

$$
T_n(x) = \sum_{k = 0}^{n} a_k x^k 
$$


Esto es, entonces, una aproximación por polinomios. 


\subsubsection{Una mejor aproximación}

Una mejor aproximación se podría obtener si expandimos la clase de funciones a las funciones racionales:

$$
G_{[n/m]} = \frac{P_n(x)}{Q_m(x)}
$$

\newpage 

La diferencia entre la aproximación y la función aproximada (el error) es:


\begin{align*}
	err(x) &= f(x) - \frac{P_n(x)}{Q_m(x)}  \\
	\lim_{\sigma \to \infty} T_{\sigma}(x) &= f(x) \\
	\\
	err(x) 
	&= \lim_{\sigma \to \infty} T_{\sigma}(x) - \frac{P_n(x)}{Q_m(x)}   	
\end{align*}

Definamos $N = m + n $:

\begin{align*}
	err(x, \alpha) 
	&= \lim_{\sigma \to \infty} T_{\sigma}(x) - \frac{P_n(x)}{Q_m(x)}	\\
	&= \frac{ Q_m(x) \displaystyle{\lim_{\sigma \to \infty} T_{\sigma}(x)} - P_n(x)}{Q_m(x)} \\
\end{align*}


Y si queremos $err(x, N) = 0$, necesitamos:

$$ Q_m(x)\lim_{\sigma \to \infty} T_{\sigma}(x) = P_n(x) $$

\begin{align*}
	P_n(x) 
	&= \sum_{k = 0}^{n} p_k x^k \\
	Q_m(x) 
	&= \sum_{k = 0}^{m} q_k x^k 
\end{align*}



\subsubsection{Coeficientes}

El problema se puede reducir a un sistemas de ecuaciones:

\begin{align*}
	&q_0 = 1 \\
	&p_0 = a_0 \\
	&p_1 = a_1 + a_0 q_1 \\
	&p_2 = a_2 + a_0 q_2 + a_1 q_1 \\
	&p_3 = a_3 + a_0 q_3 + a_1 q_2 + a_2 q_1\\
	&\dots   \\
	&p_k = \sum_{i=0}^{k} a_i q_{k-i}
\end{align*}


\subsection{Resolución}

En nuestro caso, $n = 2$ y $m = 1$, por ende $N = 3$ y buscamos resolver:

\begin{align*}
	&p_0 = a_0 \\
	&p_1 = a_1 + a_0 q_1 \\
	&p_2 = a_2 + a_1 q_1 \\
	&0 ~ = a_3 + a_2 q_1
\end{align*}


Esto nos da que:
\begin{align*}
	&q_0 = 1 ~ &(\textit{por def.}) \\
	&q_1 = - \frac{a_3}{a_2} \\
	&p_0 = a_0 \\
	&p_1 = a_1 - \frac{ a_0 a_3 }{ a_2 } \\
	&p_2 = a_2 - \frac{ a_1 a_3 }{a_2}
\end{align*}



Dado que la función por aproximar es:

$$ D(t) = e^{-\alpha t} + \beta sin(t) $$

Los primeros coeficientes de la serie de Taylor son:

\begin{align*}
	&a_0 = D(t=t_0)  = e^{-\alpha t_0} + \beta \sin(t_0) \\
	&a_1 = \frac{\partial_t D(t=t_0)}{1!} 
	= -\alpha e^{-\alpha t_0} + \beta \cos(t_0) \\
	&a_2 = \frac{\partial_t^2 D(t=t_0)}{2!}
	= \frac{\alpha^2 e^{-\alpha t_0} - \beta \sin(t_0)}{2}
	 \\
	&a_3 = \frac{\partial_t^3 D(t=t_0)}{3!}
	= \frac{-\alpha^3 e^{-\alpha t0} - \beta \cos(t_0)}{6}	
\end{align*}

y el aproximante es (con $ \Delta t = t - t0$): 
$$
\frac{p_0 + p_1 \Delta t + p_2 (\Delta t)^2} { 1 + q_1 \Delta t } = \\
\frac{
	a_0 + (a_1 - \frac{ a_0 a_3 }{ a_2 })(\Delta t) + (a_2 - \frac{ a_1 a_3 }{a_2})(\Delta t)^2
}{
1 - \frac{a_3}{a_2}(\Delta t)
} 
$$


