\section{Método de Newton}

\subsection{Problema}

Resuelva ese sistema de ecuaciones utilizando el método de Newton partiendo del punto $(\alpha_0 , \beta_0) = (0,0)$ con una precisión para cada una de las variables de $10^{-8}$.

\subsection{Análisis}

\paragraph{Residuos}

Primero empezaremos generalizando el resultado anterior. 
Tomemos el conjunto de parámetros como $ P = \{\alpha, \beta\}$, lo cual da que:

\begin{align*}
	\partial_{P_j} \text{SSR} 
	&= 2 \sum_{k} r_k \frac{\partial r_k}{\partial P_j} \\
	&= 2 \sum_{k} - r_k \frac{ \partial f(x_k)}{\partial P_j} \\
\end{align*}

donde $P_j$ $\epsilon$ P $= \{\alpha, \beta\}$.

\paragraph{Iteración}

El paso fundamental del algoritmo es:

\begin{equation}
	\textbf{J}^T\textbf{J} \Delta \text{P} = \textbf{J}^T  \textbf{r}
\end{equation}

donde:

\begin{equation*}
	\textbf{r} = [r_1 \dots r_n ]
\end{equation*}

\paragraph{Jacobiano}

Ahora pasamos a construir el Jacobiano, dado por:

\begin{equation}
	J_{ij} = - \frac{\partial r_i}{\partial P_j} = \frac{\partial f(x_i)}{\partial P_j} 
\end{equation}


Lo cual da que:

\begin{equation}
	\frac{ \partial \text{SSR} }{ \partial P_j } = -2 \sum_i r_i J_{ij}
\end{equation}


\subsection{Resolución}

\paragraph{Algoritmo}

El algoritmo para resolver el sistema es Gauss-Newton. Si vemos el apartado anterior, podemos replantear:

\begin{equation*}
	\textbf{J}^T\textbf{J} \Delta \text{P} = \textbf{J}^T  \textbf{r}
\end{equation*}

con:

\begin{align*}
	\textbf{A} \Delta P &= \textbf{b} \\
	\textbf{A} &= \textbf{J}^T\textbf{J} \\
	\textbf{b} &= \textbf{J}^T  \textbf{r}
\end{align*}

\begin{itemize}
	\item \text{Iniciamos con los parámetros} = [0.0, 0.0]
	\item \text{Calculamos el cambio de parámetro.}
	\item \text{Si el cambio es menor que el límite, paramos.} 
\end{itemize}

\paragraph{Valor} Después de ejecutar el código (que se puede ver en \ref{code:ex3}), nos da que los parámetros que minimizan el error son:

$$\alpha = 0.42952981942913554$$

$$\beta = -1.9842186638017174$$